\chapter{Genetic Algorithms}

\begin{chapquote}{Richard Dawkins}
``It is absolutely safe to say that if you meet somebody who claims
not to believe in [what Richard Dawkins claims to believe], that
person is ignorant, stupid or insane.''
\end{chapquote}

In the previous chapter, we defined what it means for a
proposition to be satisfied under a given valuation.
We also said informally, what it means for a propositional
logic formula to be satisfiable -- namely, that there exists
a valuation such that the formula is satisfied.

This formulation is rather easy to comprehend, but in practice
the amount of checks that we may need to perform to see if
a formula is satisfiable may grow exponentially with the number
of atomic propositions occurring in the formula: adding another
atomic proposition doubles the number of possibilities, so the
problem quickly becomes intractable.

The problem isn't just theoretical. Although propositional
calculus is extremely simple, it has some useful practical
applications in software verification.

Besides, despite the fact that the general solutions fail
for more complex formulas, one can frequently find some
\textit{heuristics} that perform very well for the most
common cases.

\section{Biological inspiration}

One of such heuristics is called \textit{genetic algorithms}.
It is inspired by the process of natural selection. Most
generally, the idea is to encode the solution as a sequence
of \textit{chromosomes} that behave as binary switches, where
each switch splits the search space in half.

The solutions are searched in larger groups called
\textit{populations}. Each specimen of a population
is evaluated using a so-called \textit{fitness function},
which measures the quality of a solution. Rather than
deciding that a given specimen is either good or bad,
the fitness function assigns it a grade.

\subsection{Chromosomal crossover}

During the evolution process, the best solutions are
combined with each other, and the worst ones are rejected.
The hope is that if we combine a few individuals into
a new one, it has a chance to inherit the good traits
of its originals. In practice, the chromosomes
of couples of the best specimens from the population
are recombined in a \textit{crossing-over} process:
therefore the new individuals can be regarded as
an offspring of the old ones.

The crossing-over can be defined as a function that takes
two chromosomes, splits them at a random point and returns
a new one composed of the prefix of the first one and
the suffix of the second:

\begin{Snippet}
(define (cross-over daddy mommy)
  (assert (= (length daddy) (length mommy)))
  (let* ((position (random (length daddy)))
	 (sperm (take daddy position))
	 (ovum (drop mommy position)))
    `(,@sperm ,@ovum)))
\end{Snippet}

Of the things that deserve attention, the first one is the
presence of an \textit{assertion}. An assertion essentially
does nothing more than informing the reader about the
assumptions that the author made about the intended usage
of a function -- in this particular case, he assumed that
the two arguments -- \texttt{mommy} and \texttt{daddy}
-- are lists of equal length.

Then we choose a random\footnote{Each occurrence of the
\texttt{(random n)} expression can be replaced with
a value randomly chosen from the range $0$ to $n-1$.
This \textit{indeterminism} is the main reason why
I prefer the term \textit{expression-based programming}
to \textit{functional programming}. Disallowing such
non-deterministic expressions would make many simple
things complex.} position from the range between
$0$ and \textit{the length of chromosome} minus $1$.
We use the prefix of the \texttt{daddy} chromosome (for
example, if \texttt{daddy} is \texttt{(a\,b\,c\,d\,e)}, then
\texttt{(take daddy 3)} evaluates to \texttt{(a\,b\,c)})
and the suffix of the \texttt{mommy} chromosome (similarly,
if \texttt{mommy} is \texttt{(u\,v\,w\,x\,y)}, then
\texttt{(drop mommy 3)} evaluates to \texttt{(x\,y)}).

Finally, we construct a new list from the prefix and the
suffix.

Now that we've seen the intimate details of copulation,
we can have a look at the process from the social perspective.

\subsection{The ceremony of procreation}

Whether a given specimen becomes a daddy or a mommy is
a matter of luck, and it may happen that the same specimen
serves as a mommy during an intercourse with one partner,
and as a daddy with another. It is important however, that the
specimen with the high social status (i.e. better fitness
function) procreate more often than the ragtag.

Also, contrary to what many catholic priests say, it is
perfectly fine when a specimen copulates with itself:
when it happens, its child is an exact copy of its sole
parent (however in practice this privilege is reserved
to the elite).

The ceremony proceeds as follows. First, a census is prepared,
where each member of the population is assigned a value
of a fitness function. Then, the census is sorted according
to that value. Subsequently, the lottery drawing occurs, where each
member of the population draws an opportunity to copulate.
A single specimen can appear on the list more than once, and
the probability of being enrolled to the list is higher for
the specimens that are higher on the list. The length of
the list equals the population. 

Finally, each one of those lucky beggars is assigned a partner
from another list generated in the same way (so, as it was said
before, it can happen that the he-and-she spends the night with
a \textit{very} familiar company), and then the new generation
comes to the fore.

This cycle can repeat for a few hundreds of times or more, until
the best man from the population is satisfactory to its
transcendent creator.

It can also happen, say, once in a generation on average,
that a chromosome of a specimen is mutated.

The social perspective on the ceremony of copulation can
be expressed in Scheme in the following way:

\begin{Snippet}
(define (procreate population social-status)
  (let* ((census (map (lambda (specimen)
			`(,(social-status specimen) . ,specimen))
		      population))
	 (social-ladder (sort census (lambda ((a . _) (b . _)) 
                                       (> a b))))
	 (population (map (lambda ((status . specimen)) specimen)
			  social-ladder))
	 (size (length population))
	 (males (biased-random-indices size))
	 (females (shuffle (biased-random-indices size)))
	 (offspring (map (lambda (man woman)
			   (cross-over (list-ref population man)
				       (list-ref population woman)))
			 males females)))
     (map (on-average-once-in size (mutate #;using not)) offspring)))
\end{Snippet}

Note that the arguments to \texttt{lambda} provided to the \texttt{sort}
function are destructured (pattern-matched) in place. This is a convenient
syntactic extension to Scheme that replaces the built-in \texttt{lambda}.

Also, we provided three arguments to \texttt{map}, rather than two
-- one function of two arguments and two lists. The value of the expression
will be a list whose elements are values of the function for the first element
from the first list and the first element of the second list, the
second element of the first list and the second element of the second
list, and so on. It is assumed that the length of both lists is the
same -- and that will also be the length of the result.

We will learn how to define functions of variable number of arguments
later. For now it is sufficient to know that it is possible to do so,
which can often be convenient.

It is apparent that the \texttt{cross-over} function is given two
arguments that result from the invocation of the built-in \texttt{list-ref}
function, which takes a list of length $n>0$ and an index $i$ (starting
from $0$) and returns the $i$-th element of the list.

The procedure \texttt{biased-random-indices} generates a list of
indices with the desired distribution:

\begin{Snippet}
(define (biased-random-indices size)
  (if (= size 0)
      '()
      `(,(random size) . ,(biased-random-indices (- size 1)))))
\end{Snippet}

The length of the generated list will be \texttt{size}.
Furthermore, the index $0$ is guaranteed to appear on the list
as the last element, has a $50\%$ chance to appear as the
penultimate element, one third to appear one before, and so
on, so it will almost certainly appear on the list more than
once. On the other hand, the last index can appear only
as the first element of the list, but this opportunity
is shared evenly by all the other indices, so it is rather
unlikely that it will ever happen. This strategy is called
``equal opportunity policy'' by the social ideologies.

Two separate lists are created for \texttt{males}
and \texttt{females}. In order to increase the variation
of the population, the list of \texttt{females} is shuffled,
which gives them a chance to advance to the upper class.

The shuffling proceeds as follows: if the list has more
than one element, then a random pivot point is chosen,
and the (recursively shuffled) sub-lists are swapped with
a $50\%$ probability. Otherwise the shuffled list is
identical with the original one.

\begin{Snippet}
(define (shuffle l)
  (match (length l)
    (0 '())
    (1 l)
    (n (let ((left right (split-at l (random n))))
	 (if (= (random 2) 1)
	     `(,@(shuffle right) ,@(shuffle left))
	     `(,@(shuffle left) ,@(shuffle right)))))))
\end{Snippet}

The last thing that requires an explanation is the mutation.
We wish it to occur on average once per generation, therefore
we define a function that applies a given function with
a probability $\frac{1}{n}$ for a given $n$, where $n$ is
a positive integer:

\begin{Snippet}
(define ((on-average-once-in n action) arg)
  (assert (and (integer? n) (> n 0)))
  (if (= (random n) 0)
      (action arg)
      arg))
\end{Snippet}

The \texttt{action} of \texttt{mutation} that we wish to apply
inverts the random chromosome:

\begin{Snippet}
(define ((mutate how) specimen)
  (let* ((n (random (length specimen)))
	 (mutation (how (list-ref specimen n))))
    (alter #;element-number n #;in specimen #;with mutation)))
\end{Snippet}

The meaning \texttt{(alter n l value)} is a list that contains
all the same elements as \texttt{l}, except its \texttt{n}-th
element, which has a new \texttt{value}.

\subsection{Evolution}

The struggles of a sole generation in the quest for the perfect
society are unlikely to bring satisfactory results (especially
if these struggles boil down to copulation). Only within
the span of many lifetimes can the true value arise. It is
entirely up to the creator to decide many cycles of lives
and deaths will the world witness, how complex can its inhabitants
be, and how many of them will be brought to existence.

\begin{Snippet}
(define (evolve population #;towards criterion #;for iterations)
  (assert (and (integer? iterations) (>= iterations 0)))
  (if (<= iterations 0)
      population
      (evolve (procreate population criterion)
              #;towards criterion 
              #;for (- iterations 1))))
\end{Snippet}
\begin{Snippet}
(define (generate-specimen dimension)
  (generate-list dimension (lambda () (= (random 2) 0))))
\end{Snippet}
\begin{Snippet}
(define (generate-population size dimension)
  (generate-list size (lambda () (generate-specimen dimension))))
\end{Snippet}
\begin{Snippet}
(define (optimize dimension population-size iterations criterion)
  (let* ((population (generate-population population-size dimension))
	 (modern-society (evolve population #;towards criterion 
				 #;for iterations)))
    (argmax criterion modern-society)))
\end{Snippet}

The \texttt{generate-list} procedure takes a \texttt{lambda} expression
of no arguments that evaluates nondeterministically to some value
(in our case, it is either logical truth or logical falsehood)
and generates a list containing a specified number of results of
such evaluation:
\begin{Snippet}
(define (generate-list n generator)
  (assert (and (integer? size) (>= size 0)))
  (if (= n 0)
      '()
      `(,(generator) . ,(generate-list (- n 1) generator))))
\end{Snippet}

Lastly, the \texttt{argmax} is a library function that takes
a measure function and a list and evaluates to the element of the
list that has the greatest measure.

\section{Solving the SAT}

In the previous section, we have presented a heuristic
for optimizing certain problems with reasonable (linear)
computational means. In this section, we will apply that
heuristic to the problem of satisfiability in propositional
logic.

\subsection{Parsing DIMACS CNF files}

We are not going to operate in vacuum -- there are many
resources on the Web that provide the information and databases
of examples. However, the format in which the examples are
encoded may vary. One such format, proposed by the Center
for Discrete Mathematics and Theoretical Computer Science,
is called DIMACS CNF (for \textit{Conjunction Normal Form}),
and is encoded in text files.

A DIMACS CNF file is line-based. If a line begins with the \texttt{c}
character, then it is a comment and shall be ignored. A first
line of the DIMACS CNF file that is not a comment should take the
form:

\texttt{p cnf <variables> <clauses>}

where \texttt{<variables>} specifies the maximum number of atomic
propositions that appear in the formula, and \texttt{<clauses>}
specifies the number of disjunctive clauses in the formula, or
the number of meaningful lines that follow the given line.

After the header, there appear \texttt{<clauses>} lines that
contain sequences of integers (possibly negative) separated with
white space and terminated with zero. A number $n$ means that
an atomic formula appears in a given disjunctive clause,
or -- if it is negative -- that its negation appears in the
disjunctive clause.

This may sound complicated, so it can be instructive to see
an example. The file

\begin{Snippet}
c simple_v3_c2.cnf
c
p cnf 3 2
1 -3 0
2 3 -1 0
\end{Snippet}

would correspond to the formula 

\texttt{(and (or x1 (not x3)) (or x2 x3 (not x1)))}

As we can see, the whole expression is a conjunction (this is
why it's called \textit{conjunctive} normal form), and its
sub-expressions are disjunctions. There are two of them,
as there are lines below the DIMACS CNF header. The first
line contains numbers \texttt{1 -3 0}, which correspond
to the expression \texttt{(or x1 (not x3))}, and the second
line contains numbers \texttt{2 3 -1 0} that correspond to
the expression \texttt{(or x2 x3 (not x1))}.

It can be puzzling why someone decided to be able to represent
only conjunctions of disjunctions of formulas and their negations,
but it turns out that for every formula of classical logic
one can find an equivalent conjunctive normal form. The details
are beyond the scope of this pamphlet.

We therefore need to find a way to transform the numbers
given in the lines of the DIMACS CNF file into S-expressions.

We will be using the \texttt{(ice-9 regex)} library bundled 
with Guile and its ability to process \textit{regular expressions}.
The regular expressions will be exposed here rather than explained,
as the more detailed explanation can be found rather easily
among the vast resources of the Worldwide Web.

Also, we are going to process the file line by line. The
Scheme itself doesn't provide facilities for doing that,
but Guile comes with the \texttt{(ice-9 rdelim)} library
which provides a function that returns a next available line
from a given \texttt{input-port}. We will fix the
\texttt{input-port} to mean \texttt{(current-input-port)},
which is a notion similar to \textit{standard input} known
from UNIX and its descendants.

We shall gather the input lines from the list. The
function terminates when it approaches an \textit{end-of-file
object}:

\begin{Snippet}
(define (input-lines)
  (let* ((line (read-line (current-input-port))))
    (if (eof-object? line)
	'()
	`(,line . ,(input-lines)))))
\end{Snippet}

We shall ignore all the lines that contain anything else
than (possibly negative) numbers separated with white-spaces
and terminated with 0. For that, we will use the \texttt{filter}
function which takes a predicate and a list, and returns
a new list that contains only those elements from the old
list that satisfy the predicate (preserving the order).
Thus, the expression

\begin{Snippet}
(filter (lambda (line)
          (string-match "^\\s*(-?[0-9]+\\s+)+0\\s*$" line))
        (input-lines))
\end{Snippet}

will evaluate to a list that contains only the meaningful
lines. Note that we used the \texttt{string-match} procedure
here, that takes a regular expression and a string (I
didn't mention it before, but Scheme also provides strings as
its elementary data type) and evaluates to truth-ish value
if the string matches the regular expression\footnote{
We said that the logical truth is expressed using the
\texttt{\#t} syntax, but in fact any value other than
\texttt{\#f} is regarded as true in the context of the
\texttt{if} form and its derivatives.}. The meaning
of the regular expression is as follows. The \texttt{\string^}
character anchors the regular expression at the beginning
of the string. Then, \texttt{"\char`\\\char`\\s*"} allows a sequence
of white spaces to appear at the beginning. The group
\texttt{"(-?[0-9]+\char`\\\char`\\s+)+"} requires that there
will appear (one or more times) a non-empty sequence of digits followed
by a non-empty sequence of white-spaces, and that it can be
prepended with the minus sign. Lastly, the expression
\texttt{"0\char`\\\char`\\s*\$"} says that \texttt{0} has to appear
at the end of the sequence, and that it can optionally be
followed with a sequence of white characters.

We now have a sequence of valid lines that need to be
converted to propositional logic expressions. If a line has
a form of a string \texttt{"2 3 -1 0"}, then if we split
the line by spaces, we obtain a list of strings
\texttt{("2" "3" "-1" "0")}. We can then further
convert each of those strings (but the last one) to number
using the built-in \texttt{string->number} function. We used
the plural form, so it will probably require the use
of \texttt{map}.

Lastly, we need to convert the number to an expression,
that is -- either a symbol or a negation of a symbol.
We can therefore define:

\begin{Snippet}
(define (number->expression number)
  (if (negative? number)
    `(not ,(number->expression (- number)))
     ((numbered-symbol 'x) number)))
\end{Snippet}

where \texttt{numbered-symbol} could be defined in
terms of the primitive Scheme functions:

\begin{Snippet}
(define ((numbered-symbol symbol) number)
  (symbol-append symbol (string->symbol (number->string number))))
\end{Snippet}

Putting it all together, we can define our DIMACS-CNF
processor:

\begin{Snippet}
(define (process-dimacs-cnf)
  (define (disjunction line)
    (let* (((strings ... "0") (filter (lambda (s)
                                        (not (equal? s "")))
                                      (string-split line #\space)))
           (numbers (map string->number strings))
           (expressions (map number->expression numbers)))
      `(or . ,expressions)))
  (let* ((data-lines (filter (lambda (line)
			       (string-match 
                                "^\\s*(-?[0-9]+\\s+)+0$" 
                                line))
			     (input-lines))))
    `(and . ,(map disjunction data-lines))))
\end{Snippet}

The only thing that can be unobvious is binding the value
of the expression \texttt{(string-split line \#\char`\\space)}
to the pattern \texttt{(strings ... "0")}. What it actually
does is that it skips the last element (that is assumed to
be \texttt{"0"}) from the result of \texttt{string-split}.
The \texttt{...} operator for the pattern matcher behaves
in a complementary way to the \texttt{unquote-splicing}
operator, so for example, in the expression
\begin{Snippet}
(match '(1 2 3 4 5 6)
  ((a b c ... z))
  `(,a ,b ,@c ,z))
\end{Snippet}
\texttt{a},\texttt{b} and \texttt{z} will be bound with
\texttt{1}, \texttt{2} and \texttt{6}, respectively, and the
\texttt{c} name will be bound with the list \texttt{(3 4 5)},
so the value of the above expression will simply be the list
\texttt{(1 2 3 4 5 6)}. The \texttt{...} operator cannot
appear more than once in a list, so for example patterns like
\texttt{(a ... k ... z)} are illegal, as they would result
in an ambiguous match. The \texttt{...} operator has
another magical property -- in the pattern
\begin{Snippet}
(match '((a . 1) (b . 2) (c . 3))
  (((keys . values) ...)
   `(,keys ,values)))
\end{Snippet}
the \texttt{keys} symbol will be bound with the list \texttt{(a b c)},
and the \texttt{values} symbol -- with \texttt{(1 2 3)}, so the
list subjected to pattern matching will be \textit{unzipped}.
We will be making a use of this feature later.

Back to our parser, note that we used the value
\texttt{\#\char`\\space} as the second argument to 
\texttt{string-split}. It is a Scheme name for the ``space'' character 
(characters are also a data type in Scheme). Note also
that we had to apply another filter on the result of the
\texttt{string-split}, in order to remove empty strings from it,
that appeared if there were two or more
consecutive spaces in a given line.

\subsection{Applying the genetic strategy}

Now that we know how to convert the DIMACS CNF files to Scheme,
we can test whether the provided formulas are satisfiable -- or,
to put it in another way -- seek for a valuation under which
they are satisfied.

It is rather obvious that our chromosomes will denote the
values of subsequent atomic propositions for the valuations
(therefore, they will be rather straightforward to decode).
We need to know how many distinct atomic formulas are there
in the main formula:

\begin{Snippet}
(define (atomic-formulas proposition)
  (match proposition
    (('not clause)
     (atomic-formulas clause))
    ((operator . clauses)
     (delete-duplicates (append-map atomic-formulas clauses)))
    ((? symbol?)
     `(,proposition))))
\end{Snippet}

The code uses two library functions, \texttt{delete-duplicates}
(that does exactly what it says\footnote{
It is very important in programming to choose appropriate
names for concepts, whether ultimate or intermediate, because
knowing that we can trust the names, we don't need to resort
to documentation.
}) and \texttt{append-map}, which
is like \texttt{map} except that its functional argument is
expected to return a list, and that list is \textit{appended}
to the resulting list, rather than \textit{inserted} as
a single element.

The structure of the formulas -- the conjunctive normal
form -- also prompts us with the fitness function: that
will be the number of the satisfied disjunctive sub-clauses.

\begin{Snippet}
(define (number-of-satisfied-subformulas #;of cnf #;for chromosome)
  (let* ((variables (generate-variables #;from 1
                                        #;to (length chromosome)))
         (valuation (map cons variables chromosome))
         (('and . or-clauses) cnf))
    (count (lambda (subformula)
             (satisfied? subformula #;under valuation))
           or-clauses)))
\end{Snippet}
\begin{Snippet}
;; where
(define (generate-variables #;from first #;to last)
  (if (> first last)
    '()
    `(,((numbered-symbol 'x) first) 
      . ,(generate-variables #;from (+ first 1) #;to last))))
\end{Snippet}

Here, we use the library function \texttt{count} that takes
a predicate and a list and returns the number of elements
of the list that satisfy the predicate.

The only thing that's left is to put the pieces together
and enjoy the show\footnote{The file ``dubois20.cnf'',
among others, was obtained from the website
\url{http://people.sc.fsu.edu/~jburkardt/data/cnf/cnf.html}.}:

\begin{Snippet}
(let* ((formula (with-input-from-file "dubois20.cnf" 
                  process-dimacs-cnf))
       (dimension (length (atomic-formulas formula)))
       (measure (lambda (chromosome)
                  (number-of-satisfied-subformulas #;of formula
                                               #;for chromosome))))
  (optimize dimension 160 100 measure))
\end{Snippet}
