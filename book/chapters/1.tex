%%%%%%%%%%%%%%%%
% NEW CHAPTER! %
%%%%%%%%%%%%%%%%
\chapter{Introduction}

\begin{chapquote}{Hal Abelson and Gerald Sussman, 
\textit{Structure and Interpretation of Computer programs}}
\begin{quotation}
  ``The acts of the mind [...] are chiefly these three:
  1. Combining several simple ideas into one, and thus all complex
  ideas are made.
  2. [...] bringing two ideas, whether simple or complex, together,
  and setting them by one so as to take a view of them at once,
  without uniting them into one, by which it gets all its ideas
  of relations.
  3. [...] separating them from all the other ideas that accompany
  them in their real existence: this is called abstraction, and thus
  all its general ideas are made.''
\end{quotation}

  John Locke, \textit{An Essay Concerning Human Understanding} (1690)

  [...]

  A powerful programming language [...] serves as a framework within 
  which we organize our ideas [...]. Every programming language has
  three mechanisms for accomplishing this:
  \begin{itemize}
  \item \textbf{primitive expressions} [...]
  \item \textbf{means of combination} [...]
  \item \textbf{means of abstraction} [...]
  \end{itemize}

\end{chapquote}

\section{Epistemic role of programming}

This text was written with the belief that programming
is not only a way of automating common repetitive tasks,
but -- most importantly -- it is a means (probably the most
efficient) of representing knowledge and understanding
the world and its phenomena, as well as expressing original
ideas.

This view is frequently obscured by some bad practices
that are widespread among programmers. Apart from expressing
knowledge and ideas, programming languages allow programmers
to do virtually everything they like -- it is therefore
possible that they write code which gets in the way to
understanding, rather than facilitate it\footnote{Apparently
even excellent programmers and great erudites have been failing 
at this task. One of the most extreme examples is the code from 
Donald Knuth's \textit{The Art of Computer Programming}.
Written in the \texttt{MIX} assembly, the code only hides the
conceptual structure of presented solutions from the reader.}.

The claim that many programmers write code in ``the wrong way'',
and that the author of this pamphlet knows which way is 
``the right way'' may seem pretentious. It must therefore
inevitably be left to the reader to judge whether the author's
claims are true.

\subsection{Expression-based programming}

Throughout this pamphlet, we will be perceiving computer programs
as constructions of compound objects. This view has recently been
gaining more popularity, but it is still far from the mainstream
view, which claims that a program is a set of instructions that
ought to be performed in a given order.

The latter view is often called \textit{imperative programming},
and has traditionally been contrasted with \textit{declarative
 programming}.
Within the realm of declarative programming, there is a particular
subset called \textit{functional programming}, where the programmer
creates programs by defining functions (in mathematical sense).

However, since the notion of functional programming has recently
been somewhat appropriated by Haskell, and for some reasons that
shall become clear later, I will prefer to use the term
\textit{expression-based programming} or \textit{constructive
programming}.

For example, let's say that someone wishes to write a program that
computes \textbf{a sum of squares of initial seven prime numbers}. 
The imperative program that does this could look as follows:

%\lstset{language=Python}
\begin{Snippet}

counter := 7
number := 0
sum := 0
while(counter > 0):
    if is_prime(number):
        sum := sum + number^2
        counter := counter - 1
    number := number + 1

\end{Snippet}
%\lstset{language=Lisp}

After its execution, the \texttt{sum} variable will contain the
desired value.

However, if we take a look at the specification of the problem,
we will notice that it can be itself perceived as a part of the
solution: the expression ``sum of squares of initial seven prime
 numbers'' has a certain linguistic (or syntactic) structure,
and a corresponding conceptual (or semantic) structure. The 
syntactic structure may not be immediately apparent, but simple
analysis shows that the outer form of the expression is 
``sum of $X$'', where $X$ is ``squares of initial seven prime
numbers''. We can further decompose the latter, extracting the 
form ``squares of $Y$'',  where $Y$ is ``initial seven prime 
numbers''. Making yet another step, we obtain the expression
``initial $N$ prime numbers'', where $N$ is ``seven''.

\begin{center}
\Tree [.sum 
  [.squares 
    [.prime-numbers initial seven ] ] ]
\end{center}

This isn't the only way in which we can perform decomposition.
That's because the natural language that we use in our everyday
communication is naturally ambiguous. For instance, by 
``initial seven prime numbers'', do we mean ``a set of initial
 prime numbers'', or ``a sequence of initial prime numbers''?

This distinction is of course insignificant from the point of view of
our task, because addition is commutative, and hence the order in which
we compute this sum is arbitrary. On the other hand, if we asked
someone to name the initial seven prime numbers, it would be likely
that he or she would enumerate them in the ascending order. If we
investigated further into that phenomenon, it would probably turn
out that the order is a consequence of the order in which the numbers
are computed by the speaker. Indeed, what a shocking experience it
would be to hear those numbers being uttered in a random or descending
order!

In order to avoid such shocks, people apply the so-called
\textit{cooperative principles} to their conversations\footnote{
The theory of Cooperative Principles was formulated by Paul Grice.
He explicated four maxims that ought to be applied in effective
communication. Probably the most relevant to programming and
the one that is least respected by programmers is the Maxim
of Manner, which recommends to avoid obscurity of expression,
avoid ambiguity, be brief and be orderly.}: they
try to adopt their utterances to the expectations of their
interlocutors (to the extent in which they can recognize them),
or -- in the parlance of programmers -- they ``try to design
clean and simple APIs''\footnote{Seriously.}.

Note however, that even if there is no natural order of elements
in a set, whenever we wish to enumerate elements from that set,
imposing \textit{some} order is inevitable.

Therefore, unless specified otherwise, whenever we speak about
some multitude of elements, we shall mean a sequence of those
elements, rather than a set (even if the order is irrelevant).

Returning to our example, we have a function ``initial $N$
prime numbers'', where $N$ is a natural number. The value
of a function, for a given $N$, is an ascending sequence
of $N$ initial prime numbers. For instance, the value (or meaning
\footnote{As it was shown by Frege, the word ``meaning'' is also
ambiguous, and can either mean \textit{intension/connotation} or 
\textit{extension/denotation}.
Intension of a semantically compound concept is its formulation
in simpler terms.}, or denotation) of expression ``initial 7 
prime numbers'' is a sequence \texttt{(2 3 5 7 11 13 17)}
\footnote{Of course, we cannot write down a sequence, and so we are 
forced to write down a \textit{textual representation} of a sequence.
In fact, we can only write down two distinct textual expressions,
hoping that the reader will establish in her or his mind, that those
distinct two expressions are meant to refer to the same abstract
object.}.

This is as far as we can go with the linguistic analysis.
Obviously, the expressions under consideration consist of
words, but their meanings do not (at least not in this
particular case)\footnote{It is possible to build linguistic
expressions whose meanings consist of words. A fine example
is a linguistic expression ``This sentence'', which refers
to a sequence of two words: \texttt{(This sentence)}.}:
the sequence of numbers consists of numbers and their order.
We could therefore ask: what is the structure of a sequence?

We can note that sequences can be empty, as it is the case
with the denotation of the expression ``initial 0 prime numbers'',
or ``empty sequence''. Non-empty sequences can be decomposed
into their first element, and the remainder of its elements
(which may be empty). This prompts us with a recipe to construct
sequences recursively. First, let's note that the phrase
``initial $N$ prime numbers'' is synonymous with ``$N$ least
prime numbers greater than 0''. Therefore we have function
of two arguments: ``$N$ least prime numbers greater than $M$''.

We can define this function as follows:

\begin{quotation}
The meaning of ``$0$ least prime numbers greater than $M$''
is an empty sequence.

The meaning of ``$N+1$ prime numbers greater than $M$''
is:
\begin{itemize}
\item a sequence whose first element is $M+1$, and whose
remaining elements are the $N$ prime numbers greater
than $M+1$, \textbf{if} $M+1$ is a prime number;
\item otherwise it's just the same as $N+1$ prime numbers
greater than $M+1$.
\end{itemize}
\end{quotation}

Although the above formulation is a legitimate description
that explains how to construct a sequence of subsequent
prime numbers, it disconcertingly brings legal gibberish
to mind. It also makes it difficult to distinguish between
the strict definitional part of the text from its more
explanatory fragments. Moreover, as it was noted before,
the syntactical structure of the text isn't immediately
clear even for the simplest examples.

In order to handle these issues, it is customary to
introduce a formal notation. It is a common practice
in mathematics, where mathematicians use special symbols
to denote certain abstract concepts, and in logic, where
logicians try to make certain reasonings more explicit.

\section{Introduction to Scheme}

In the quest for our perfect notation, we shall stick
to the rule of parsimony: we want to make as few assumptions
and conventions as possible, but we also want to make sure
that they are universal. Firstly, let's embrace the
compound syntactical units in parentheses. Then, the
expression ``sum of squares of initial seven prime numbers''
becomes:

\texttt{(sum (squares (initial seven prime numbers)))}

While at first the parentheses may seem difficult
to match, they allow to avoid confusions and ambiguities
that are typical for the natural language. Note also,
that we got rid of the ``of'' prepositions, because
they provide no information: we can simply read \texttt{(f x)}
as ``$f$ of $x$'' (provided that $f$ has only one argument).

Furthermore, let's stick to a convention, that a single
concept is expressed using a single word, where by words
we mean sequences of letters separated with parentheses
or white spaces. Also, let's make sure that the ruling
word of the expressions appears at the first position.
We can make compound words using hyphens:

\texttt{(sum (squares (prime-numbers initial seven)))}

This formulation expresses the structure of our problem
well enough using so-called \textit{fully parenthesized
Polish notation} -- certainly we can no longer be confused
with lawyers. Now the only thing that's left is to explain
in basic terms what we mean by \texttt{sum}, \texttt{squares}
and \texttt{prime-numbers}. The question which terms
are to be regarded as basic depends on the context.
In this particular case, we also happen to have written
(a fragment of) a program in a programming language called Scheme,
which enumerates a whole set of notions that are regarded as 
simple\footnote{The exact specification of the basic terms
of Scheme is provided in the \textit{Revised$^n$ Report on the
Algorithmic Language Scheme} that can be found at
\url{http://www.schemers.org/Documents/Standards}.}, and the notion of
\texttt{prime-numbers} certainly does not belong to that set.
It does, however, contain the notions of numbers, addition,
multiplication, division, comparison and it also provides some means
to construct and deconstruct sequences. Most importantly,
it provides means for defining new notions.

\subsection{Constructions}

Paraphrasing our construction of ``least $N$ prime numbers 
greater than $M$'' from previous section, we can write:

\begin{Snippet}
(define (prime-numbers amount from)
  (if (= amount 0)
    '()
    (if (prime? from)
      (cons from (prime-numbers (- amount 1) (+ from 1)))
      (prime-numbers amount (+ from 1)))))
\end{Snippet}

Although conceptually we present nothing new here compared to
the previous section, there are a few formal elements that need
to be explained. First of all, we use a special \texttt{define}
form from the Scheme language. This way, we give a name
to the new concept that we are defining\footnote{Although the
concept might not be so new for us, because we have defined
it informally in the previous section, it certainly is new for
the Scheme interpreter.}. We also observe that there are
quite a few new concepts used in the definition. The body (or
\textit{definiens}, if you please) consists of an \texttt{if}
expression, which -- in general -- takes the following form:

\texttt{(if <condition> <then> <else>)}

The value of the \texttt{if} expression depends on the value
of \texttt{<condition>}. If it is true in a given context, then
the value of the whole expression becomes the value of the 
\texttt{<then>} expression. Otherwise, it is the value of the
\texttt{<else>} expression. (This is actually quite straightforward,
if you think about it.)

The expression that appears as the \texttt{<condition>} has 
a form \texttt{(= amount 0)}. The meaning of the word \texttt{=} is
the \textit{numerical comparison}, commonly found in mathematics. It
can be surprising to see it used as \texttt{(= x y)} rather
than \texttt{(x = y)}. Recall however our convention that
``the ruling word of the expression appears at the first
position''. We could depart from that rule for mathematical
expressions (and whoever knows where else), but this would
only make our rule set more complex. However, it actually
turns out that mathematical expressions do not appear so
frequently in computer programs to make it worth the trouble,
and having a simple language turns out much greater an
advantage.

As one can expect, the \texttt{(= amount 0)} expression evaluates
to the logical truth, if the meaning of \texttt{amount} is 0 in that
context, and otherwise it evaluates to the logical falsehood.

In the first case, the value of the \texttt{if} expression
becomes \texttt{'()}, that is -- an empty sequence. In the
second case, the expression becomes:

\begin{Snippet}
(if (prime? from)
  (cons from (prime-numbers (- amount 1) (+ from 1)))
  (prime-numbers amount (+ from 1)))
\end{Snippet}

This, again, is an \texttt{if} expression. Its \texttt{<condition>}
clause takes form \texttt{(prime? from)}. The intent is, that
\texttt{prime?} is a function that evaluates to the logical
truth if its argument is a prime number, and to the logical
falsehood if its argument is a non-prime number. We will explain
later what we exactly mean by that.

It may be surprising for someone to see a punctuation mark
being a part of a name of a concept. There is however a common
convention among Scheme programmers to end the names of
\textit{predicates}\footnote{i.e. functions whose values can either
be logical truth or logical falsehood.} with question marks\footnote{
A cautious reader probably noticed that the \texttt{=} function
used before is also a predicate, although it's name doesn't end
with a question mark.
This lack of consequence isn't bothersome in practice though, because
naming conventions are not as essential to the structure of a language
as the rules given earlier.}.

The \texttt{<then>} clause of the considered \texttt{if} expression
is interesting. It has a form:

\texttt{(cons from (prime-numbers (- amount 1) (+ from 1)))}

There are four things to be noted here: first, the expression
refers to the \texttt{cons} function, which is used for
\textit{constructing} sequences. Second, the second argument
to \texttt{cons} refers to the notion of \texttt{prime-numbers},
which is currently being defined.
This form of definitions is often called \textit{recursive}. Third,
we use the elementary mathematical operations of \textit{addition}
and \textit{subtraction} in the same way that we used the numerical
comparison earlier, i.e. using the prefix notation. Lastly, the
fact that we pass \texttt{amount} decreased by one to the recursive
call, in conjunction with the fact that the sequence of 0 prime
numbers is empty, guarantees that our construction will terminate,
provided there is enough prime numbers.

The \texttt{<else>} clause of the second \texttt{if} expression
has a form:

\texttt{(prime-numbers amount (+ from 1))}

In other words, it establishes the fact that if $n$ isn't
a prime number, then the prime numbers starting from $n$ are
the same as the prime numbers starting from $n+1$.

So far we have defined \texttt{prime-numbers} in terms of primitive
notions, such as addition, subtraction, construction, conditionals
and recursion. The only thing that's left is to explain what it means
that a number is \texttt{prime?}.

We know from the mathematics that a prime number is a number whose
only natural divisors are 1 and the number itself. We also know
that a number cannot be divisible by a number greater than itself.
Hence, in order to find a list of numbers, we can try all numbers
starting from 1.

\begin{Snippet}
(define (divisors n from)
  (if (> from n)
    '()
    (if (= (remainder n from) 0)
      (cons from (divisors n (+ from 1)))
      (divisors n (+ from 1)))))
\end{Snippet}

This definition has a very similar structure to the previous one.
We used some new elementary functions, namely the comparison predicate
\texttt{>} (greater than) and the \texttt{remainder} operation.

Perhaps the program could be made clearer if we had defined what it
means for a number to divide another number. However, if we decided
to define a predicate \texttt{(divides? a b)}, it wouldn't be clear
whether we mean that ``\texttt{a} divides \texttt{b}'' or that
``\texttt{b} divides \texttt{a}'' -- this is an apparent limitation
of the prefix notation.

This problem could be solved by adding another pair of parentheses
in our definition:

\begin{Snippet}
(define ((divides? a) b)
  (= (remainder a b) 0))
\end{Snippet}

Now it should be clearer that we ask whether \texttt{b} divides
\texttt{a}.
Although the core Scheme doesn't allow such notation, one of
the modules contained in this pamphlet's repository legalizes
it. From the semantic point of view, it defines \texttt{divides?}
as a function of \texttt{a} whose value is a function of \texttt{b}
that compares the remainder of the division of \texttt{a} by
\texttt{b} with \texttt{0}.

Since we have means to construct the sequence of divisors of a given
number, we can check whether the number is a prime simply by comparing
its list of divisors with a list containing only the number 1 and
the considered number itself:

\begin{Snippet}
(define (prime? number)
  (equal? (divisors number 1) (list 1 number)))
\end{Snippet}

We have used here two new primitive functions: \texttt{equal?} and
\texttt{list}. They are not entirely primitive, as they could be
defined rather easily in terms of other operations (we will see how
to do that shortly), but every Scheme implementation certainly
provides those, so for now we can consider them as such.

The \texttt{list} function simply returns a list of values of
its arguments. In this particular example, we could as well have written
\texttt{(cons 1 (cons number '()))} instead of using \texttt{list}.

The \texttt{equal?} function is a predicate which says that two
or more compound objects are identical.

In Scheme, there are a few distinct concepts of equality. The first
that we have seen is called \textit{numerical equality} and is
expressed using the \texttt{=} symbol. The second that we've just
seen above is called \textit{identity}, and is expressed using
the \texttt{equal?} symbol.
The third is called \textit{sameness} and is expressed using
the \texttt{eq?} symbol.

It is guaranteed that two symbols having the same shape are
\texttt{eq?} and that all empty sequences are \texttt{eq?}. However
for various reasons it is not guaranteed that two instances of
the same number are \texttt{eq?}.
Two lists probably won't be \texttt{eq?} even if they have
all the same elements in the same order -- the reason for that
is that usually each usage of \texttt{cons} or \texttt{list}
creates a new object in the computer memory.

It is noteworthy that if two things are \texttt{eq?} or \texttt{=},
they will certainly be \texttt{equal?}, so the latter predicate is
most general and can be used in most situations.

\subsection{How do we know}

So far, we have defined a few notions that were used in the
formulation of the original problem. We defined them by means
of recursion, so we had to worry only about two things:
what would be the shape of the simplest instance of a given
notion, and how do we build a more complex instance out of
a simpler one.

It would be helpful to have some means of verifying whether
our definitions actually work as expected. It turns out, that
those means are also very simple: those are \textit{substitution}
and \textit{reduction}.

Let's see how it works for the \texttt{divisors} function.
To begin with something simple, let's see what is the denotation
of \texttt{(divisors 4 1)}. We obtain the \textit{connotation}
of the expression by substituting the complex notion of
\texttt{divisors} with its definition, replacing any formal
parameters with their values. Thus we obtain:

\begin{Snippet}
(if (> 1 4)
  '()
  (if (= (remainder 4 1) 0)
    (cons 1 (divisors 4 (+ 1 1)))
    (divisors n (+ 1 1))))
\end{Snippet}

We see that the first condition is clearly false, because $1$
is smaller than $4$. We can therefore substitute the
conditional with its \texttt{<else>} branch:

\begin{Snippet}
(if (= (remainder 4 1) 0)
  (cons 1 (divisors 4 (+ 1 1)))
  (divisors n (+ 1 1))))
\end{Snippet}

If we substitute the expression \texttt{(remainder 4 1)} with
its value, i.e. the number $0$, it is apparent that the
condition \texttt{(= 0 0)} is true, so we can replace the
conditional with its \texttt{<then>} branch:

\begin{Snippet}
(cons 1 (divisors 4 (+ 1 1)))
\end{Snippet}

It is also clear that \texttt{(+ 1 1)} is $2$. We therefore
go back to the initial situation, with the difference that
our expression is now an argument to \texttt{cons} function,
and its \texttt{from} argument has value $2$ instead of $1$.

By expanding the expression with its connotation and performing
appropriate reductions, we obtain the following form:

\begin{Snippet}
(cons 1 (cons 2 (divisors 4 3)))
\end{Snippet}

In next iteration of substitutions and reductions it will
turn out that the remainder of $4$ divided by $3$ is non-zero,
so the above expression can be rewritten as:

\begin{Snippet}
(cons 1 (cons 2 (divisors 4 4)))


\end{Snippet}

As the remainder of the division of $4$ by $4$ is 0, we
rewrite the above as:

\begin{Snippet}
(cons 1 (cons 2 (cons 4 (divisors 4 5)))
\end{Snippet}

Now the \texttt{<condition>} of the outer \texttt{if} form
is satisfied, so the \texttt{(divisors 4 5)} expression evaluate
to empty sequence to yield:

\begin{Snippet}
(cons 1 (cons 2 (cons 4 '()))
\end{Snippet}

which eventually evaluates to the sequence \texttt{(1 2 4)}.
The assumptions under which we performed the above inference
are called \textit{the substitutional model of computation}.

\subsection{Destructuring}

So far, we managed to explain what we mean by the expression 
``initial $N$ prime numbers'' from our formulation of the original 
problem. The formulation was a bit more complex, because it asked
to sum the squares of those numbers.

So in order to accomplish that task, we still need to explain what
we mean by squares and their sum. Let's begin with the denotation
of the expression ``squares of $Y$''. From our previous
considerations we know that $Y$ is a list of numbers. Intuitively, 
if $Y$ were a list \texttt{(1 2 3)}, then the meaning of the outer
expression would be a list \texttt{(1 4 9)}, that is,
each element of the denotation of the expression ``squares of $Y$''
is a square of the corresponding element of $Y$.

A square of a number is -- by definition -- that number multiplied
by itself:

\begin{Snippet}
(define (square number)
  (* number number))
\end{Snippet}

To make it clear what the squares of a sequence of numbers are, or
rather to explain how they are constructed, we need to notice that
the squares of an empty sequence of numbers are an empty sequence.

As for a non-empty sequence, we know that it can be decomposed into
its first element and the sequence of its remaining elements. We
could therefore define squares of a non-empty sequence as a sequence,
whose first element is a square of the first element of the input
sequence, and whose remaining elements are the squares of the
remaining elements of the input sequence (this statement sounds
very tautological indeed, but we will see shortly how to deal
with that)

\begin{Snippet}
(define (squares numbers)
  (if (eq? numbers '())
    '()
    (cons (square (car numbers)) (squares (cdr numbers)))))
\end{Snippet}

The most important thing about this definition is that it reveals
the means for destructuring lists. For historical reasons, the
function whose value is the first item of a list is called
\texttt{car}, and the function that yields the remaining of
the list is called \texttt{cdr}.

It should be clear that for any values \texttt{a} and \texttt{b},
\texttt{(car (cons a b))} is \texttt{a} and \texttt{(cdr (cons a b))}
is \texttt{b}.

The names \texttt{car} and \texttt{cdr} used to be popular, because
they prompted contractions like \texttt{(caddr x)}, which was an
abbreviation of \texttt{(car (cdr (cdr x)))}. Although some examples
from this chapter will be written using those functions, a more
powerful (and readable) technique will be presented in one of
the later sections.

Looking back at our definition of \texttt{squares}, we notice
that there's something essentially wrong with it. Linguistically,
the word \textit{squares} is just a plural form of \textit{square}.
This relation isn't reflected in our usage.

Suppose that we wanted to solve a problem of finding the cubes
of a list of numbers, rather than squares. Following the above
pattern, we would need to define the function \texttt{cube} that
would explain what a cube of a number is. Then we would define
a function \texttt{cubes} that would apply the function \texttt{cube}
to each elements of a given list.

Such a practice would be very unhandy. We would rather wish to
have a function, let's call it \texttt{plural}, that would take
a function of one argument, and return a function of list of
arguments:

\begin{Snippet}
(define ((plural concept) list)
  (if (eq? list '())
    '()
    (cons (concept (car list)) ((plural concept) (cdr list)))))
\end{Snippet}

While it would be completely fine to have and use such function,
historically things have gone differently, and the grammatical plural
form is achieved using the \texttt{map} function\footnote{Note that
\texttt{map} is a built-in function, so there's no actual need to
define it. Moreover, the built-in version is in fact better than
our variant, so redefining it is strongly discouraged.}:

\begin{Snippet}
(define (map f l)
  (if (eq? l '())
    '()
    (cons (f (car l)) (map f (cdr l)))))
\end{Snippet}

Therefore, ``the squares of initial seven primes'' should be written
as \texttt{(map square (prime-numbers 7 1))}.

The only thing that's left is to explain what the sum of a list is.
We should consider the boundary condition first: what would be the
sum of an empty list? We know that no elements are zero elements,
so we could expect that a sum of zero elements will be zero. Otherwise
it's the value of the first element added to the value of the
remaining elements\footnote{We could also define \texttt{sum}
without resorting to zero, by specifying it for lists whose length
is at least one, but that would actually make the code more
complicated.}:

\begin{Snippet}
(define (sum numbers)
  (if (eq? numbers '())
    0
    (+ (car numbers) (sum (cdr numbers)))))
\end{Snippet}

We can now present the solution to the original problem: ``the sum
of squares of initial 7 primes'' is

\texttt{(sum (map square (prime-numbers 7 1)))}

Note the similarity between the formulation and the solution. This
sort of similarity isn't reserved for the Scheme programming language,
and can be achieved in virtually every programming language that
allows to create functions\footnote{Some additional practical
considerations may regard languages that do not automatically reclaim
storage, such as C.}. The advantage of Scheme however is its
simplicity:
we already know almost everything about the syntax of our language,
and our knowledge of its semantics is sufficient to construct many
advanced programs.

The most important thing is that, unlike the imperative code
given earlier, this code explains exactly what it is about: the
reader doesn't need to recognize any implicit patterns, as
everything has been laid out explicitly.

The only thing that may seem mysterious at first sight is the
role of arguments to \texttt{prime-numbers}. We could add a comment
that would explain that role.

Scheme has three types of comments. There are comments that begin
with a semicolon (\texttt{;}) and end with the end of the line,
block comments that begin with \texttt{\#|} and end with \texttt{|\#},
and very useful comments that begin with \texttt{\#;} and comment out
exactly one expression that follows them. We could therefore
enrich our solution with additional information:

\texttt{(sum (map square (prime-numbers 7 \#;greater-than-or-equal 1)))}

\subsection{Getting your hands dirty}

So far, we have been using a formal notation to express certain
mathematical notions in a precise manner -- everything was going
on on the paper and in our heads. There's yet another important
advantage of our notation: it can be ``comprehended'' by computers,
so on one hand, we have a practical instance of verification whether
our beliefs regarding the meanings of certain expressions are right,
and on the other -- we can employ the computer to do some parts
of reasoning for us. Of course, if you do not wish an inferior machine
to perform the noble task of reasoning for you, you can still
benefit from reading the pamphlet and exercising the strict notation.

There are many free practical implementations of Scheme. Probably
the most popular and accessible is Racket, which comes with
a very intuitive editor. Another nice implementation is
MIT/GNU Scheme, that comes with a slightly less intuitive editor.
There are implementations that focus on speed, such as Gambit
or Stalin, and those that compile to some popular platforms like Java
or JavaScript. There's also an implementation called Biwa Scheme,
that can be run directly in a web browser.

The implementation that is going to be used in this text is
called Guile. It is not so easy to install, but it is very
convenient for practical interactions and incremental development.
It integrates particularly well with the Emacs editor through
the Geiser package. The only bad thing about Emacs is that it requires
getting used to.

The detailed information regarding setting up Emacs, Geiser and Guile
is platform-dependent and as such is beyond the scope of this pamphlet.
An interested reader will easily find the required information on the
web.

Assuming that everything is up and running, you can evaluate
a single expression in a Scheme interpreter. The most common way
is to enter an expression into a so-called \textit{command line}
or REPL (which stands for ``read-eval-print loop''), however
having a proper environment like Emacs with Geiser, it is possible
to enter an expression in a text file and evaluate it using
a magic keystroke\footnote{By default, it's \texttt{C-x C-e}
in Geiser.}.

Before we get anything done, we need to legalize the syntactic
extensions that are used throughout this pamphlet: first, the
additional modules need to be downloaded from the pamphlet's
repository, available at \url{https://github.com/panicz/pamphlet}
(the \texttt{libraries} directory), and placed somewhere
in Guile's \texttt{\%load-path}, and then we need to type in
the expression:
\begin{Snippet}
(use-modules (ice-9 nice-9) (srfi srfi-1) (pamphlet))
\end{Snippet}

Finally, we can feed the interpreter with all the necessary
definitions: we need to explain what we mean by \texttt{divisors},
what it means for a number to be \texttt{prime?}, how to construct
a sequence of \texttt{prime-numbers}, what is a \texttt{square}
of a number and how to \texttt{sum} a list of numbers.

Then, we can ask the computer to give us the value of the
expression \texttt{(sum (map square (prime-numbers 7 1)))}.

\subsection{Looking down}

So far, as we were referring to certain expressions of the English
language whose meanings we were trying to paraphrase, we were usually
quoting them. For example, we could say: ``The meaning
of the expression <<first three natural numbers>> is a sequence
\texttt{(0 1 2)}''. It is therefore typical, especially when
talking about the meanings of linguistic expressions, to use
the \textit{quotation} operator.

In Scheme, the quotation operator is expressed using
\texttt{quote} form. For example, the value of the expression
\texttt{(quote (sum (map square (prime\--numbers 7 1))))}
is a sequence of two elements, whose first element is
a symbol \texttt{sum}, and whose second element is a sequence
of three elements, whose first element is a symbol \texttt{map},
second element is a symbol \texttt{square}, and third element
is a sequence of three elements: a symbol \texttt{prime-numbers},
a number $7$ and a number $0$.

We've actually already seen an example of quotations in our
definitions: it turns out that the operation of quotation is
so useful that it has a notational shorthand: \texttt{'\textit{x}}
means the same as \texttt{(quote \textit{x})}, so our notation
for the empty sequence -- \texttt{'()} -- could equivalently
be written as \texttt{(quote ())}. Note that the \texttt{quote}
operator is very different from the quotation marks that are
often used in programming languages for constructing strings,
because it allows us to quote structures, rather than create
flat sequences of characters.

As you might have already noticed, the compound expressions
in Scheme are also sequences\footnote{Note that we need to quote
the empty sequence in order for it to mean the empty sequence,
because the meaning of an empty expression is unspecified.
Some dialects of Lisp blur this distinction and assume that
the meaning of an empty expression is the empty sequence.
This may be confusing though, and the Schemers are proud that
their language is conservative in this regard.}.
This property, called \textit{homoiconicity},
is one of the most valuable properties of Scheme, because it
allows to treat programs as syntactic objects and write programs
that transform other programs in Scheme rather easily.

It also allows to create new languages with a common syntax.
Or, as Alfred Tarski would put it, it allows to use Scheme
as a \textit{meta-language} for some other formal languages\footnote{
It is also possible to use Scheme as a meta-language for itself
-- this technique, called \textit{meta-circular evaluation},
has been explored in the grand book \textit{Structure and
Interpretation of Computer Programs}\cite{SICP}. This idea was
at the heart of the seminal paper \textit{Recursive Functions
of Symbolic Expressions and their Computation by Machine}
by John McCarthy\cite{McCarthy1960}, which gave birth to
the predecessor of Scheme called Lisp.}.

To illustrate that idea, we will describe the semantics of
a simple formal system called \textit{propositional logic}, which
captures how compound sentences are built from simpler ones.
The simplest units of propositional logic are called
\textit{atomic formulas} -- they stand for propositions
that aren't analyzed in simpler terms, and that can only
be \textit{asserted} or \textit{rejected}.

The expressions of propositional logic are either propositional
variables $p,q,r,...$ standing for atomic
formulas, or junctions of expressions: $\phi \vee \psi$,
$\phi \wedge \psi$, $\phi \Rightarrow \psi$, $\phi \equiv \psi$,
$\neg \phi$. These compound expressions mean disjunction (``$\phi$ or
$\psi$''), conjunction (``$\phi$ and $\psi$''), implication
(``if $\phi$ then $\psi$''), equivalence (``$\phi$ if and only
if $\psi$''), and negation (``it is not the case that $\phi$''),
accordingly. For the remainder of this example we shall be concerned
only with conjunction, disjunction and negation.

The semantics of propositional logic determines the logical value
(i.e. truth or falsehood) of each formula with respect to some given
\textit{valuation}, i.e. a mapping from propositional variables to
logical values.

If a formula is atomic, then its value is simply specified by the
valuation. Otherwise it is a compound formula: a negation, a disjunction
or a conjunction. If it is a negation, then it is true only if
the negated formula is false; if it is a disjunction, then it
is true only if at least one of its sub-formulas is true; finally,
if it is a conjunction, then it is true only if all its sub-formulas
are true.

A formula is satisfiable if there exists a valuation under which
it is true. A formula is logically valid if it is true for
any valuation. For example, $p \vee \neg p$ is a valid formula,
because if $p$ is true, then the whole disjunction is true, and
if $p$ is false, then $\neg p$ is true and the disjunction is
true as well. It is easy to see that if a formula is valid then
its negation is not satisfiable.

Testing whether a given valuation satisfies a given formula
can be a tedious task that can be performed more efficiently
on a computer.

We will use our fully parenthesized Polish notation to represent
formulas of propositional calculus. We will write
\texttt{(and $\phi$ $\psi$)} to mean conjunction,
\texttt{(or $\phi$ $\psi$)} to mean disjunction and
\texttt{(not $\phi$)} to mean negation.

Note that conjunction and disjunction both possess an algebraic
property of \textit{associativity}, i.e. 
$(\phi \circ \psi) \circ \chi = \phi \circ (\psi \circ \chi)$ for
$\circ \in \{ \wedge, \vee \}$. We can therefore interpret
\texttt{(and $\phi_1$ $\phi_2$ ... $\phi_n$)} as
$\phi_1 \wedge \phi_2 \wedge ... \wedge \phi_n$ with no ambiguity.
The same applies to disjunction.

Having a fixed representation, we can explicate the conditions
under which a formula is \texttt{satisfied?} under a given valuation.
However, we need to find some way to represent a valuation.
One obvious representation of a mapping
is by enumerating the name-value pairs. For example, if there
are three distinct atomic formulas $p,q,r$ in $\phi$, one of
the eight possible valuations could be written as
\texttt{'((p . \#t)(q . \#f)(r . \#t))}. The \texttt{\#t} and
\texttt{\#f} inscriptions represent the logical truth and
falsehood in Scheme.

Notice the strange dot that appears between each key and its
corresponding value. This is because we said that we will
represent the mapping using a key-value \textit{pairs}, rather
than two-element lists. We could as well have chosen the
latter representation, but we didn't, so that the strange
dot could appear.

We said earlier that \texttt{cons} is a primitive function
that constructs lists. However, it would be more accurate
to say that \texttt{(cons a b)} creates (or \textit{allocates})
a new pair, whose \textit{left value} (or \texttt{car})
is its first argument, \texttt{a}, and whose \textit{right value}
(or \texttt{cdr}) is its second argument, \texttt{b}.

A list is therefore either an empty sequence, or a pair
whose right value is a list\footnote{A nesting of pairs
whose last right value is not the empty list is called
\textit{improper list}.}.

Now it turns out that the list \texttt{(1 2 4)} could
equivalently be written as \texttt{(1 . (2 . (4 . ())))}\footnote{
Note the spaces between the symbols and the dot: they are important,
because it is fine to use the dot as a part of a symbol or a number.},
and similarly, the \texttt{(cons a b)}
expression could be written as \texttt{(cons . (a . (b . ())))}.
The list notation is therefore a shorthand for a pair
notation\footnote{Note that the distinction of the left element
of the pair as a first element of the list is only a matter
of convention -- for example, if Lisp was invented in the Arabic
countries, it could have been the other way around.}.

So under this representation, if we wish to obtain a value
of a given atomic formula from a given valuation, we may need
to use the \texttt{lookup} procedure:

\begin{Snippet}
(define (lookup key #;in mapping)
  (let* ((this (car mapping))
         (remaining (cdr mapping))
         (name (car this))
         (value (cdr this)))
    (if (eq? name key)
      value
      (lookup key remaining))))
\end{Snippet}

We see a new construct here -- the \texttt{let*} form.
The form is useful for naming the intermediate components
of the object that we are trying to obtain. We could equivalently
have written:

\begin{Snippet}
(define (lookup key #;in mapping)
  (if (eq? (car (car mapping)) key)
    (cdr (car mapping))
    (lookup key (cdr mapping))))
\end{Snippet}

but this only increases the nesting level of expressions and makes
the code more difficult to read. We could also have introduced
intermediate names using local \texttt{define} forms:

\begin{Snippet}
(define (lookup key #;in mapping)
  (define this (car mapping))
  (define remaining (cdr mapping))
  (define name (car this))
  (define value (cdr this))
  (if (eq? name key)
    value
    (lookup key remaining)))
\end{Snippet}

The extent of the inner \texttt{define} forms is limited
to the outer \texttt{define} form. However, the structure
of the code is more difficult to follow, as it contains
many obscure symbols.

Finally, I promised before that we won't be using the
\texttt{car} and \texttt{cdr} functions, because our code
doesn't deal with vehicles nor compact disc recording. Instead
we can use the marvelous feature of \textit{pattern matching}:

\begin{Snippet}
(define (lookup key #;in mapping)
  (let* ((((name . value) . remaining) mapping))
    (if (eq? name key)
      value
      (lookup key remaining))))
\end{Snippet}

This definition is much more compact than the original one,
but may be slightly more difficult to follow. It simply
decomposes the structure of \texttt{mapping}, i.e. that
it is a pair whose left value is a pair, and names
its parts accordingly: the left value of the left value
of \texttt{mapping} is named \texttt{name}, the right
value of the left value of \texttt{mapping} is named
\texttt{value}, and the right value of \texttt{mapping}
is named \texttt{remaining}.

Note that all the above variants of the \texttt{lookup}
function assume that there has to be an element whose
left value is \texttt{key} somewhere in the mapping,
and it is an error if that condition isn't satisfied
(because it is an error to talk about the left or right
value of something that isn't a pair).

Equipped with the \texttt{lookup}, we can go back
to the original problem and explain what it means
for a proposition to be satisfied:

\begin{Snippet}
(define (satisfied? formula #;under valuation)
  (match formula
    (('and . clauses)
     (every (lambda (clause)
              (satisfied? clause #;under valuation))
            clauses))
    (('or . clauses)
     (any (lambda (clause)
            (satisfied? clause #;under valuation))
          clauses))
    (('not clause)
     (not (satisfied? clause #;under valuation)))
    ((? symbol?)
     (lookup formula #;in valuation))))
\end{Snippet}

There are many new things going on in here: the
main part of the definition is the \texttt{match}
expression, which -- in general -- takes the following
form:

\begin{Snippet}
(match <expression> 
  (<pattern-1> <value-1>) 
  (<pattern-2> <value-2>)
  ...)
\end{Snippet}

The first \texttt{<pattern>} is \texttt{('and . clauses)}.
It matches against all pairs whose left element is the symbol
\texttt{and}, and if the match succeeds, the right element
of the pair gets bound with the name \texttt{clauses}.
So this part of code could be rewritten as\footnote{Technically
speaking, the code is equivalent only if the \texttt{formula}
always evaluates to the same value. In other cases, we would
need to capture the value of \texttt{formula} before expansions.}:
\begin{Snippet}
(if (and (pair? formula) (eq? (car formula) 'and))
  (let* ((clauses (cdr formula)))
    <value-1>)
  (match formula
    ((<pattern-2> <value-2>)
     ...)))
\end{Snippet}

The \texttt{<value>} part from the scope of the first
\texttt{<pattern>} is
\texttt{(every (lambda (clause) (satisfied? clause valuation)) clauses)}.
The function \texttt{every} takes a predicate and a list, and
evaluates to the logical truth if the predicate is satisfied
for every element in the list.

The first argument to \texttt{every} is a \texttt{lambda}\footnote{
One has to admit that \texttt{lambda} is a rather strange name.
The reason for this weirdness is that the semantics of Scheme
is based on a very simple powerful model of computation known
as $\lambda$-calculus. Allegedly, its name is a result of editor's
mistake, and dates back to the logical \textit{opus magnum}
of Bertrand Russell and Alfred North Whitehead, \textit{Principia
Mathematica}, where they used the notation $f(\hat{x})$
to mean a function of $x$, rather than the value of $f$ in the
point $x$. Alonzo Church modified the notation by taking the
$\hat{x}$ before the function application, i.e. $\hat{x} . f(x)$.
A typesetter wrote that as $\Lambda x . f(x)$, someone read that
as the Greek letter \textit{lambda}\cite{Harrison1996}.} expression.
The \texttt{lambda} expression takes the form:

\texttt{(lambda <arguments> <body>)}

It creates a new anonymous function. Actually we've already used
\texttt{lambda} expressions before, albeit implicitly: the
\texttt{(define (f . <args>) <body>)} is equivalent to
\texttt{(define f (lambda <args> <body>))}.

The second \texttt{(<pattern> <value>)} is similar, except that
it uses the \texttt{any} function. The only unobvious part that is
left is the \texttt{(? symbol?)} pattern. This pattern matches
only if the \texttt{formula} satisfies the predicate \texttt{symbol?},
or in other words, the pattern is matched if it is the case that
\texttt{(symbol? formula)}.

\subsection{Building structures}

The \texttt{match} expression isn't a native Scheme form. It is
a ``specialized language'' for destructuring complex objects, and
it was added to Scheme through its powerful syntax extension
mechanism.

So far we have been using the \texttt{cons} or \texttt{list}
functions for creating complex structures. I promised that in
the remainder of this text we won't be using \texttt{car}
and \texttt{cdr}, because \texttt{match} allows to do the same
task better.

We usually won't be using \texttt{cons} as well. There's yet another
``specialized language'' that is meant for building complex objects.
It makes use of the fact, that -- just as the notation \texttt{'x}
is a shorthand for \texttt{(quote x)} -- the notation
\texttt{`x} is a shorthand for \texttt{(quasiquote x)}, 
\texttt{,x} is a shorthand for \texttt{(unquote x)}, and
\texttt{,@x} is a shorthand  for \texttt{(unquote-splicing x)}.

Those words alone, \texttt{quasiquote}, \texttt{unquote} and
\texttt{unquote-splicing}, mean nothing. As with every other word
in Scheme, you can make it mean whatever you want. However,
the \texttt{quasiquote} by default is a syntactic extension
that gives a special meaning to the other two words, and allows
to quote only a part of a list. For example, the expression
\texttt{`((+ 2 3) ,(+ 2 3))} evaluates to a list \texttt{((+ 2 3) 5)}, 
and the expression  \texttt{`(0 ,@(divisors 4 1) 5)} evaluates
to \texttt{(0 1 2 4 5)}.

So instead of writing \texttt{(cons a b)}, we shall write
\texttt{`(,a . ,b)} from now on.

\subsection{Conclusion}

At this point, we know enough about Scheme to use it to represent
advanced notions and to solve some practical problems.
