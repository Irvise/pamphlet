
\chapter{Matrix Operations}

\begin{chapquote}{David Jacobs}
``Code is like poetry; most of it shouldn't have been written.''
\end{chapquote}

In the previous chapters we've been dealing with certain methods
of \textit{soft computing}. The numerical packages (or ``programming
languages'', as some people call them) like R, SPSS or Matlab
have often been praised for their ``native'' support for matrix
operations and statistical notions.

In this chapter, we will show how easily these concepts can be
implemented in Scheme using the means that we already know. We will
also learn to define and use \textit{variadic} functions.

It is common in mathematics and science to organize the computations
in the form of \textit{matrices}, because this allows to process
large amounts of data in a systematic and uniform manner.

A \textit{matrix} is a rectangular array of numbers, for example

\begin{equation*}
  \begin{pmatrix}
    1 & 2 & 3 & 4 \\
    0 & 4 & 5 & 8 \\
    0 & 0 & 7 & 6
  \end{pmatrix}
\end{equation*}

We can use lists of lists of equal length to represent matrices,
for example

\begin{Snippet}
'((1 2 3 4)
  (0 4 5 8)
  (0 0 7 6))
\end{Snippet}

\subsection{Matrix addition}

Given a matrix, we can determine its \textit{dimensions}.
For example, the above matrix is three-by-four.

\begin{Snippet}
(define (dim x)
  (match x
    ((first . rest)
     `(,(length x) . ,(dim first)))
    (_ 
     '())))
\end{Snippet}

Probably the simplest algebraic operation that one can
conduct on a matrix is \textit{matrix addition}: given
matrices of the same dimensions, we add each element
of one matrix to the corresponding element of the
second matrix:

\begin{Snippet}
(define (M+2 A B)
  (assert (equal? (dim A) (dim B)))
  (map (lambda (a b)
	 (map + a b))
       A B))
\end{Snippet}
\begin{Snippet}
(e.g.
 (M+2 '((1 2 3)
        (4 5 6)) '((1 2 3)
                   (4 5 6))) ===> ((2  4  6)
                                   (8 10 12)))
\end{Snippet}

\subsection{Variadic functions}

I called the function \texttt{M+2} to emphasize that it
takes two arguments. However, because matrix addition is
associative (like conjunction and disjunction in propositional
logic), so it might be more convenient to write
\texttt{(M+ A B C)} instead of \texttt{(M+ (M+ A B) C)}
or \texttt{(M+ A (M+ B C))}. Actually you can check that
this is already the case for scalar addition: the
expression \texttt{(+ 1 2 3)} evaluates to $6$. This
is also true for multiplication.

We can use the \texttt{M+2} function to perform the
generalization to the arbitrary number of arguments,
but no less than one.

\begin{Snippet}
(define (M+ . MM)
  (match MM
    ((M)
     M)
    ((A B . X)
     (apply M+ `(,(M+2 A B) . X)))))
\end{Snippet}

We have used an \textit{improper list} of arguments,
or a \textit{dotted tail notation}. An improper list
is a list whose last \texttt{cdr} is not the empty
list. If the list of arguments is improper, then the
symbol in the dotted tail will be bound the list of
remaining arguments, so for example if we had no
\texttt{list} function defined, we could use that
feature:

\begin{Snippet}
(define (list . items)
  items)
\end{Snippet}
\begin{Snippet}
;; or equivalently:
\end{Snippet}
\begin{Snippet}
(define list (lambda items items))
\end{Snippet}

Note the use of the \texttt{apply} function. It takes
a function and a list of arguments, and \textit{applies}
the function to arguments. For example, if \texttt{l}
is a list containing three elements, then
\texttt{(apply f l)} is equivalent to
\texttt{(f (first l) (second l) (third l))}. However,
if the number of arguments is unknown, then the use
of \texttt{apply} is inevitable if we want to preserve
generality.

Note also, that since the \texttt{max} and \texttt{min}
functions can take an arbitrary number of arguments
(and return the greatest and the smallest of them,
respectively), instead of defining the \texttt{maximum}
and \texttt{minimum} functions, we could have written
\texttt{(apply max ...)} and \texttt{(apply min ...)}.

The \texttt{apply} function can also take a variable
number of arguments -- all arguments between the first
and the last are simply \texttt{cons}ed to the last,
so we could have written \texttt{(apply M+ (M+2 A B) X)}
in the previous definition, yielding the same effect.

\subsection{Transpose}
If we swap columns and rows of a matrix, we obtain
its \textit{transpose}. For example, the transpose
of the matrix
\begin{equation*}
  \begin{pmatrix}
    1 & 2 & 3 \\
    4 & 5 & 6 
  \end{pmatrix}
\end{equation*}
is the matrix
\begin{equation*}
  \begin{pmatrix}
    1 & 4 \\
    2 & 5 \\
    3 & 6
  \end{pmatrix}
\end{equation*}

The code for constructing the transpose of a matrix is
a bit tricky:

\begin{Snippet}
(define (transpose M)
  (apply map list M))
\end{Snippet}

To see why it works, let's try to evaluate
\begin{Snippet}
(transpose '((1 2 3)
             (4 5 6)))
\end{Snippet}

According to our definition, it can be rewritten as

\begin{Snippet}
(apply map list '((1 2 3) (4 5 6)))
\end{Snippet}

which -- according to what we said earlier -- can
be rewritten as

\begin{Snippet}
(map list '(1 2 3) '(4 5 6))
\end{Snippet}

which yields the desired result.

\subsection{Matrix multiplication}

Multiplying two matrices is much trickier than adding
them and is best explained by a math tutor pointing
fingers on the blackboard.

If $A$ and $B$ are matrices, the product $AB$ is
defined only if the number of columns of matrix $A$
equals the number of rows of the matrix $B$. Then,
the number of rows of $AB$ is the same as the number
of rows of $A$, and the number of columns of $AB$
is the same as the number of columns of $B$.

\begin{Snippet}
(define (M*2 A B)
  (assert (let* (((A-rows A-cols) (dim A))
		((B-rows B-cols) (dim B)))
            (= A-cols B-rows)))
  (let* ((B^T (transpose B)))
    (map (lambda (rA)
	   (map (lambda (cB)
		  (sum (map * rA cB)))
		B^T))
	 A)))
\end{Snippet}

We can extend the \texttt{M*2} to support arbitrary
number of arguments analogically as we did for \texttt{M+2}.

There are some other important operations that can be defined
for matrices, like the determinant or rank. Their exposition
is beyond the scope of this pamphlet, but some of them can
be found in this pamphlet's repository.

